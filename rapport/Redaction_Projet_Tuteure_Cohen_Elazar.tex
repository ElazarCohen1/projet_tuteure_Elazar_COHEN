\documentclass[12pt]{article}
\usepackage[utf8]{inputenc}
\usepackage[french]{babel}
\usepackage[margin=2.5cm]{geometry}
\usepackage[T1]{fontenc}
\usepackage{graphicx}
\usepackage{xcolor}
\usepackage{hyperref}
\hypersetup{
    colorlinks=true,
    linkcolor=blue,
    urlcolor=blue,
    citecolor=blue,
    filecolor=blue
}

\title{Projet Tuteuré: Bots culinaire discord}
\author{Elazar Cohen}
\date{\today}

\begin{document}

    \maketitle

    \tableofcontents
    \newpage

    \section{Introduction}
        Ce fichier présente mon projet tuteuré pour le deuxieme semestre de L3.
        Le projet est un bot discord dans un premier temps qui propose a un groupe de personne 
        des recettes personnalisé a chacun en tenant compte de plusieurs critères comme les aliments qu'ils n'apprecient pas, 
        qu'ils apprecient etc.


    \section{Présentation du projet et fonctionnalités}

        \paragraph{Introduction}
            Le projet consiste à concevoir un bot Discord capable de proposer des recettes personnalisées aux utilisateurs d’un même groupe.
            Les utilisateurs pourront interagir avec le bot via des commandes textuelles.
            Contrairement à un simple système de recommandation de recettes, le bot créera des recettes personnalisées à partir des informations sur les utilisateurs :
            leurs goûts, les instruments de cuisine qu’ils ont à disposition ou encore leur budget.
            Le bot est conçu pour fonctionner dans un cadre collaboratif, ou pour être adapté à une seule personne.

        \paragraph{Objectifs du projet}
            L’objectif principal de ce projet est de développer un système capable de générer des recettes culinaires adaptées aux préférences et contraintes des utilisateurs.
            Le projet vise à montrer qu’il est possible d’utiliser un modèle de langage pour produire des contenus personnalisés de manière contrôlée, en tenant compte de données utilisateurs structurées.
            Un autre objectif est de proposer un outil simple d’utilisation, intégré à une plateforme de communication existante, afin de faciliter l’adoption par les utilisateurs.

        \paragraph{Fonctionnalités principales}
            Le bot Discord proposera plusieurs fonctionnalités :
            \begin{itemize}
                \item création et gestion de profils utilisateurs (préférences alimentaires, contraintes, budget) ;
                \item génération de recettes personnalisées à partir des informations fournies ;
                \item adaptation des recettes à un utilisateur unique ou à un groupe ;
                \item prise en compte de contraintes simples comme les aliments non appréciés ou indisponibles ;
                \item interaction avec les utilisateurs via des commandes textuelles.
            \end{itemize}


        \section{Analyse des besoins}

            L’utilisation du bot répond à un besoin courant : proposer des idées de repas adaptées à des personnes ayant des goûts et des contraintes différentes.
            Dans un contexte de groupe, il est souvent difficile de trouver une recette qui convienne à tout le monde.
            Le bot vise à simplifier cette prise de décision en générant automatiquement des propositions cohérentes avec les préférences exprimées.

            Les besoins principaux identifiés sont :
            \begin{itemize}
                \item une personnalisation des recettes ;
                \item une prise en compte de plusieurs utilisateurs ;
                \item une interaction simple et rapide ;
                \item une génération de recettes originales et adaptées.
            \end{itemize}

            

        \section{Choix techniques}

            \paragraph{Langage et environnement}
                Le projet sera développé en Python.
                Ce langage a été choisi pour sa simplicité, sa lisibilité et son écosystème riche, notamment dans le domaine de l’intelligence artificielle.
                Python permet également une intégration simple avec des bibliothèques dédiées à la création de bots Discord.
            \paragraph{Plateforme Discord}
                Discord a été choisi comme plateforme de déploiement car il s’agit d’un outil largement utilisé pour la communication de groupe.
                L’intégration du bot dans un serveur Discord permet une interaction naturelle avec plusieurs utilisateurs dans un même espace.

            \paragraph{Modèle de langage}
                Le bot s’appuiera sur un modèle de langage exécuté localement.
                Ce choix permet de garder un contrôle sur le fonctionnement du système et d’éviter la dépendance à des services externes.
                Le modèle sera utilisé pour générer des recettes à partir d’un contexte fourni par le système.

            
            

        \section{Conception générale du système}

            Le fonctionnement du bot repose sur plusieurs composants :
            \begin{itemize}
                \item une interface utilisateur via Discord ;
                \item un module de gestion des profils utilisateurs ;
                \item un module de génération de recettes ;
                \item un modèle de langage chargé de produire les recettes.
            \end{itemize}

            Lorsqu’un utilisateur effectue une demande, le bot récupère les informations nécessaires, construit un contexte adapté et transmet ces informations au modèle de langage afin de générer une recette personnalisée.
        
        \section{Organisation} 
            \subsection{A faire} 
                \paragraph{Semaine 1 : 12 fevrier - 19 fevrier} 
                    \begin{itemize} 
                        \item Organisation et clarification du travail demandé 
                        \item Recherche des données 
                        \item Création du shéma relationnel et création du dump de la base de donnée relationnel 
                    \end{itemize} 
                \paragraph{Semaine 2 : 19 fevrier - 26 fevrier} 
                    \begin{itemize} 
                        \item Création du bot discord 
                        \item Automatisation pour rentrer les données dans la base et vérifications des données 
                        \item Automatisation de la recherche des données par index dans la base de donnée 
                        \item Téléchargement du LLM locale 
                    \end{itemize} 
                \paragraph{Semaine 2 : 19 fevrier - 26 fevrier} 
                    \begin{itemize} 
                        \item Joindre les 3 parties (données, LLM et Bot) 
                        \item Lancement du bot pour avoir des feedbacks 
                        \item possiblement finir si je n'ai pas fini une partie 
                    \end{itemize} 
            \subsection{Fait}
            

        \section{Limites et difficultés}

            Plusieurs difficultés sont anticipées dans le cadre de ce projet :
            \begin{itemize}
                \item la gestion de contraintes contradictoires entre plusieurs utilisateurs ;
                \item la cohérence des recettes générées ;
                \item le contrôle de la personnalisation tout en conservant une certaine originalité ;
                \item les limites liées aux performances d’un modèle de langage local.
            \end{itemize}


            
        \pagebreak
        \section{Perspectives d’amélioration}

            Plusieurs pistes d’amélioration peuvent être envisagées :
            \begin{itemize}
                \item amélioration de la personnalisation à long terme ;
                \item ajout de retours utilisateurs pour ajuster les préférences ;
                \item prise en compte de nouvelles contraintes nutritionnelles ;
                \item amélioration de la diversité des recettes générées.
            \end{itemize}

            

        \section{Conclusion}

            Ce projet tuteuré a pour objectif de mettre en œuvre un bot Discord capable de générer des recettes personnalisées à l’aide d’un modèle de langage.
            Il permet d’aborder des notions telles que la personnalisation, la gestion de données utilisateurs et l’utilisation raisonnée de l’intelligence artificielle.
            Le projet constitue une application concrète des compétences acquises durant la licence informatique.

            

\end{document}